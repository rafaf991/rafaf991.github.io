



\documentclass[10pt,
 article,
 amsmath,amssymb
]{revtex4-2}



\usepackage[utf8x]{inputenc}

\usepackage[T1]{fontenc}
\usepackage{listings}
\usepackage{physics}
\usepackage{amsthm}
\usepackage{amssymb}
\usepackage{graphicx}
\usepackage[a4paper,
            bindingoffset=0.2in,
            left=1in,
            right=1in,
            top=1in,
            bottom=1in,
            footskip=.25in]{geometry}

\graphicspath{{../../../images/}}




\newtheorem{theorem}{Theorem}[section]
\newtheorem{example}[theorem]{Example}
\newtheorem{remark}[theorem]{Remark}



\newtheorem{proposition}{Proposition}
\newtheorem{corollary}{Corollary}[theorem]
\newtheorem{lemma}[theorem]{Lemma}
\newtheorem{definition}{Definition}


\newcommand{\CC}{\mathbb{C}}
\newcommand{\RR}{\mathbb{R}}
\newcommand{\LL}{\mathcal{L}}



\usepackage{tikz}

\usepackage{hyperref}


\usepackage[english]{babel}
\usepackage[english]{isodate}









\setcounter{footnote}{0}

\hypersetup{colorlinks=true,urlcolor=blue}









\begin{document}


\title{BH guide to Quantum Gravity}



\author{Rafael C.}




\date{2025-03-22}







\maketitle

\tableofcontents







This is a continuation/complement of the "Topics in gravity" notes.
\section{Spin $2-$fields}
We start from a spin $2$ field $g=\eta + h$. We saw the $E-H$ action gives $\sqrt{-g} R \sim \LL^{(2)}+ \LL^{(3)}+\dots$
the first term is the kinetic term of $h$ and the third is of the form $h \partial h \partial h$, further terms can be found by taking the $h$ stress tensor and coupling to it self. 
This will generate a theory with a new stress tensor which can be couple to itself again. Repeat $\to S_{E-H}$.

Here, derivatives means they couple to energy. This interactions introduce in general divergences 
and we need counter terms at each order as the theory is non renormalizable. At one loop $$\LL_{1lp}=\frac{1}{\epsilon}( \alpha \bar{R}^2 + \beta \bar{R}_{\mu \nu} \bar{R}^{\mu \nu} )$$
where we take $R^2$ terms so they cannot be reabsorved in the definition of $R$ and, $\varepsilon=4-D$. This gives an EFT of $QGR$

\begin{equation}
    S=\int \sqrt{-g}\qty(\Lambda + \frac{2}{\kappa} R+ c_1 R^2 +c_2 R_{\mu \nu }R^{\mu \nu }+ \dots)
\end{equation}
renormalization then gives $c_1 \to c_1+ \frac{1}{\epsilon} \alpha R^2$ and so on.
\subsection{Scattering}
Vector interactions. Graviton interactions. 

EFTs interactions get resolved by new d.o.f. (improve unitarity)


What are black holes? Thermodynamic quantities! $T\sim \frac{1}{M} \sim E^{-1} \sim \omega$ "very soft scattering" soft quanta.

Picture of Peskin book cover. At high energy some theories (asymptotically free) are good to "zoom in". GRavvitosn they become more and more complicated. Non renormalizability.


\section{CFTs vs Quantum Gravity}
Consider box of radius $R$, entropy $S= c R^{d-1} T^{D-1}$ of a conformal theory with $c$ it's d.o.f. . Also, $E_{CFT} \sim T^{D} V$ so $S_{CFT}=c (ER)^{\frac{D-1}{D}}$
but Bekenstein-Hawking entropy is $S_{BH}= \frac{A_{H}}{L_{pl}^{D-2}} \sim (E L)^{\frac{D-2}{D-3}}$
 so $S_{CFT}$ is not the same as $S_{BH}$. Quantum gravity is not a CFT.

 \subsection{Gravity in AdS}
 \begin{equation}
    ds^2=-\qty(1+ \frac{r^2}{L^2} - \frac{\mu}{r^{D-3}}) dt^2 + \frac{dr^2}{\qty(1+ \frac{r^2}{L^2} - \frac{\mu}{r^{D-3}})} +r^2 d\Omega_{D-2}^2
 \end{equation}
 if $\mu^{\frac{1}{D-3}} >> L$ then $$S\sim \frac{L_{AdS}}{L_{Planck}}^{\frac{D-2}{D-1}} (EL_{AdS})^{\frac{D-2}{D-1}}$$. Ads is like a box, then $S_{BH} \sim S_{CFT}$ in one less dimension of $AdS$.

\subsection{Thermal QFTs}
Thermal state at temperature $T=\frac{1}{\beta}$ corresponds to correlation functions in imaginary time (Wick rotated) 
$t=-it_E$ that are periodic $t_E\sim t_E+\beta$ then $$G_F^\beta(t_E,x)= G^\beta (t_E+\beta,x)$$
\begin{example}
    Finite temperature correlation functions. Consider a SHO. $S_{SHO}=\int dt \frac{1}{2}\dot{x}^2 -\frac{1}{2} \omega^2 x^2$
    $X(t)=\frac{a e^{-i\omega t}+a^\dagger e^{i\omega t}}{\sqrt{2\omega}}$ for a finite temperature the ocupation number $N= a^\dagger a$ is given by 
    $$\langle N \rangle_\beta =\frac{1}{e^{\beta \omega}-1}, \quad \beta=\frac{1}{T}$$
    whith $$\langle ... \rangle_\beta= \frac{\sum_E e^{-\beta E} \bra{E}\dots \ket{E}}{\sum_E e^{-\beta E}}$$ then harmonic oscilattor is 
    $$=\frac{\sum_{n\in \mathbb{N}} e^{-\beta \omega(n+\frac{1}{2})\bra{n} ... \ket{n}}}{\sum e^{-\beta \omega(n+\frac{1}{2}) }}$$

 The correlation function is then 
 \begin{equation}
    \langle X(t) X(0) \rangle_\beta= \frac{1}{2\omega} (\langle a a^\dagger\rangle e^{-i\omega t}+ \langle a^\dagger a \rangle e^{i\omega t})= \frac{1}{2\omega} \qty(\frac{e^{\beta \omega -i\omega t}}{e^{\beta \omega}-1} +\frac{e^{i\omega t}}{e^{\beta \omega}-1})
 \end{equation}
 setting $t\to t-i\beta$ 
 \begin{equation}
    \langle X(t-i\beta) X(0) \rangle_\beta= \langle X(-t) X(0) \rangle_\beta = \langle X(0) X(t) \rangle_\beta
 \end{equation}

 In genaral, $$\langle \mathcal{O}(t)\mathcal{O}(0)\rangle_\beta = \frac{1}{Z} \sum_E \bra{E} e^{-\beta H} \mathcal{O}(t) \mathcal{O}(0) \ket{E}= \frac{1}{2} \tr(e^{-\beta H} e^{iHt} \mathcal{O}(t) e^{-iHt}\mathcal{O}(0))  $$
 taking $t\to t-i \beta$ 
 $$\langle \mathcal{O}(t-i\beta) \mathcal{O}(0)\rangle =\frac{1}{Z} \tr(e^{iHt} \mathcal{O}(0) e^{-iHt} e^{-\beta H}\mathcal{O}(0))=\frac{1}{Z} \tr( \mathcal{O}(t) e^{-\beta H}\mathcal{O}(0)) = \langle \mathcal{O}(0) \mathcal{O}(t) \rangle_\beta$$
 
 This is known as the KMS condition.
 
 On Rindler space:
 
 $$\langle \phi(x_1) \phi(x_2)\rangle=\frac{1}{4\pi} \frac{1}{|x_1-x_2|^2}$$ but in rindler,
 $|x_1-x_2|^2=-(T_1-T_2)^2+(X_1-X_2)^2+(x_1^\perp -x_2^\perp)^2$ and, since $$T=x_R \sinh \frac{t_R}{2r_g}, \quad X=x_R \sinh \frac{t_R}{2r_g}$$, then $|x_1-x_2|^2=-(x_{R1}^2 \sinh^2 \frac{t_{R1}}{2r_g}+(x_{R1}\cosh \frac{ t_{R1}}{2r_g}-x_{R2})^2 +|x_1^\perp-x_2^\perp|^2)$
 
 then 
 \begin{equation}
     \langle \phi(t_{R1},x_{R1},x_1^\perp) \phi(t_{R2},x_{R2},x_2^\perp)\rangle=\frac{1}{4\pi |x_{12}|^2}\
 \end{equation}
 let's try $t_{R1}\to t_{R1}-i4\pi r_g$ as the RHS remains invariant, the KMS condition is satisfied:
 \begin{equation}
     \langle \phi(t_{R1},...) \phi(0,...)\rangle=   \langle \phi(t_{R1}-i 4\pi r_g,...) \phi(0,...)\rangle=
 \end{equation}
 
 More on thermal states on QFT in the Unruh effect notes. 
 
\end{example}


\section{BH thermodynamics}



\end{document}
