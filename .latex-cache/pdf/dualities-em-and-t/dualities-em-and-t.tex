



\documentclass[10pt,
 article,
 amsmath,amssymb
]{revtex4-2}



\usepackage[utf8x]{inputenc}

\usepackage[T1]{fontenc}
\usepackage{listings}
\usepackage{physics}
\usepackage{amsthm}
\usepackage{amssymb}
\usepackage{graphicx}
\usepackage[a4paper,
            bindingoffset=0.2in,
            left=1in,
            right=1in,
            top=1in,
            bottom=1in,
            footskip=.25in]{geometry}

\graphicspath{{../../../images/}}




\newtheorem{theorem}{Theorem}[section]
\newtheorem{example}[theorem]{Example}
\newtheorem{remark}[theorem]{Remark}



\newtheorem{proposition}{Proposition}
\newtheorem{corollary}{Corollary}[theorem]
\newtheorem{lemma}[theorem]{Lemma}
\newtheorem{definition}{Definition}


\newcommand{\CC}{\mathbb{C}}
\newcommand{\RR}{\mathbb{R}}
\newcommand{\LL}{\mathcal{L}}



\usepackage{tikz}

\usepackage{hyperref}


\usepackage[english]{babel}
\usepackage[english]{isodate}









\setcounter{footnote}{0}

\hypersetup{colorlinks=true,urlcolor=blue}









\begin{document}


\title{Electromagnetic and T-dualities}



\author{Rafael C.}




\date{2025-03-24}







\maketitle

\tableofcontents




\section{Electromagnetic Duality}
Consider an abelian Yang-Mills theory in the presence of an "electric" charge current to which we use the $U(1)$ connection $A$. All the allowed terms are:
\begin{equation}
    S=-\frac{1}{4g}\int F \wedge * F + F \wedge F - J_e \wedge A, \quad J_e= 2\pi i q \delta(x) 
\end{equation}
for some charge $q$ that requires Dirac charge quantization. From the equations of motion of $A$, it follows $d*F=4g J_e= 8 \pi i q \delta(x)$ so indeed
we have (also due to the Bianchi identity $dF=0$) Maxwell equations in the presence of an electric charge at $x=0$. 

Equivalently, we could have described the theory in the dual form $*F$ 
described by the local gauge connection $\tilde{A}$ such that $\tilde{F}=*F= d\tilde{A}$. To do this, we consider an action in terms of $\tilde{a}$
and $F$, where is now a general $2-$form (i.e. it might be non-closed). To recover the Maxwell equations, we introduce a lagrange multiplier for the 
constraint on $F$ being exact. The most general action is then
\begin{equation}
    S=\int -\frac{1}{4g} F \wedge * F + \frac{i}{2 \pi} d \tilde{a} \wedge F - J \wedge \tilde{a}
\end{equation}
where the second term is the lagrange multiplier and the third the "Magnetic" current. Under the equation of motion of $F$ we have $*F\sim d \tilde{a}$ 
and for $\tilde{a}$, we get $dF\sim J_m$, so indeed the roles of $*F$ and $F$ have swapped. Completing the square and, integrating out the $F$ fields, we get 

...

\section{T duality}

\subsection{Example}



\end{document}
