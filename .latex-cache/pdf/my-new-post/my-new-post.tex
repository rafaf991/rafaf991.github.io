


\documentclass{article}

\usepackage[utf8x]{inputenc}

\usepackage[T1]{fontenc}
\usepackage{listings}
\usepackage{physics}
\usepackage{amsthm}
\usepackage{amssymb}

\newtheorem{theorem}{Theorem}[section]
\newtheorem{example}[theorem]{Example}
\newtheorem{remark}[theorem]{Remark}



\newtheorem{proposition}{Proposition}
\newtheorem{corollary}{Corollary}[theorem]
\newtheorem{lemma}[theorem]{Lemma}
\newtheorem{definition}{Definition}


\newcommand{\CC}{\mathbb{C}}



\usepackage{amssymb}
\usepackage{amsmath}
\usepackage{amsthm}
\usepackage{amsfonts}

\usepackage{tikz}

\usepackage{hyperref}


\usepackage[english]{babel}
\usepackage[english]{isodate}









\setcounter{footnote}{0}

\hypersetup{colorlinks=true,urlcolor=blue}


\title{Test PDF}


\author{Rafael C.}



\date{2025-03-21}



\begin{document}

\maketitle

\tableofcontents

Fixing the gauge breaks gauge invariance on the total Lagrangian, but a new symmetry has emerged: the \textbf{BRST symmetry}. The gauge fixing term is:

\begin{displaymath}
\mathcal{L}_{g.f.} = e^{-\frac{i}{2\xi}\int f_\alpha^2(x)} = \int \mathcal{D}N e^{\frac{i \xi}{2} N \circ N - i f \circ N }
\end{displaymath}

where $f_\alpha$ is the gauge fixing term (e.g. for Lorentz gauge, $f_\alpha = \partial_\mu A^\mu_\alpha$), and $f \circ g = \int d^4x \, f(x) g(x)$.

The total Lagrangian becomes:
$\mathcal{L}_{tot} = \mathcal{L} + \mathcal{L}_{Ghosts} + \underbrace{\frac{\xi}{2} N^2(x) - f_\alpha N^\alpha}_{\mathcal{L}_{in}}$

This density is invariant under the transformation:
$\delta A = \theta D c = \theta (dc - i[A, c]), \quad \theta \text{ is a Grassmann variable.}$

And the transformation for $(\phi)$: $\delta \phi = \theta i \pi (c) \phi$

where $\pi(c)$ puts $( c )$ in the appropriate representation for $( \phi )$.

\subsection{Test s}\hypertarget{test-s}{}\label{test-s}

\begin{displaymath}
\qty(\frac{2}{3}) \CC
\end{displaymath}

dsasdas

\textbackslash{}begin\{enumerate\}
\textbackslash{}item 1
\textbackslash{}item 2
\textbackslash{}end\{enumerate\}

\begin{itemize}
\item{} 2
\item{} 2
\item{} 3
\item{} 2
\end{itemize}

This should be written in latex, and will not be put in the html version.
\{:/nomarkdown\}


\end{document}
